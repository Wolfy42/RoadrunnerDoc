\section{Systembeschreibung und -architektur}
\label{sec:system}

In diesem Abschnitt werden die in diesem Projekt verwendeten Technologien
	erläutert. Insbesondere werden die Gründe beschrieben, weswegen diese
	Technologien eingesetzt und anderen vorgezogen werden. Die Vor- sowie
	Nachteile der entsprechenden Technologien werden gegenübergestellt und
	besprochen. Zugleich werden die entsprechenden Technologien auf ihre
	Tauglichkeit in einem real logistischen Szenario geprüft.
	
\subsection{Verteiltes Datenbansystem CouchDB}
\label{subsec:couchdb}

Bei einem Transportüberwachungssystem ist die Datensicherung ein
	wichtiger Aspekt. In diesem Kapitel werden unterschiedliche 
	Möglichkeiten der Datenverwaltung betrachtet und erläutert welches
	System für das Projekt Roadrunner verwendet wird.

In diesem Projekt wurde durch die in Kapitel~\ref{sec:requirements} spezifizierten
	Anforderungen ein Fokus auf die Verteiltheit des Systems gelegt. Es fallen
	durch die mobile Transportüberwachung Daten auf mobilen Geräten an, welche
	mit einem Backendsystem synchronisiert werden müssen. Aus diesem Grund wurde
	für das Datenbanksystem kein klassisches System in Betracht gezogen. Um einen
	neuen Ansatz in der Datenpersistierung zu erlernen, wurde das verteilte
	und dokumentbasierte Datenbankmanagementsystem \emph{CouchDB}\footnote{The
	Apache CouchDB Project, \url{http://couchdb.apache.org/}} verwendet.

In einem real logistischen Szenario muss auf die Skalierbarkeit sowie die
	Robustheit von CouchDB betrachtet werden. Klassische relationale
	Datenbankmanagementsysteme bringen durch die bereits sehr gut entwickelten
	Versionen vor allem einen Vorteil in der Robustheit und Stabilität. Dennoch,
	CouchDB wird bereits seit 2005 entwickelt und liegt aktuell in der stabilen
	Version 1.1.0 (6. Juni 2011) vor und wird bereits in mehreren kommerziellen
	Projekten wie beispielsweise in Ubuntu \cite{Murphy09}[S. 1] eingesetzt.

Einen detaillierten Überblick der Verwendung von CouchDB in diesem Projekt
	wird in Kapitel~\ref{sec:couchdb} gegeben.

\subsection{Mobiles Gerät auf Basis von Android}

TODO

\subsection{Webapplikation mit dem Framework Silex}
\label{subsec:webapplication}

PHP ist eine dynamische Skriptsprache, die speziell für den Einsatz auf Webservern
	entwickelt wurde. Um ein benutzerfreundliches und einfaches Backendsystem
	für dieses Projekt zu erstellen, wurde auf mehrere bereits bestehende Frameworks
	zurückgegriffen. \emph{Silex}\footnote{vgl. \url{http://silex-project.org}} ist
	ein	Mikroframework für PHP 5.3. Es basiert wiederum auf dem Kern des
	\emph{Symfony2}\footnote{vgl. \url{http://symfony.com}} Frameworks.
	
Mit diesem Framework lassen sich einfache Webapplikationen sehr effizient in einer
	Model-View-Controller Umgebung implementieren. Durch die schöne Trennung der
	jeweiligen Schichten sowie der leichten Testbarkeit ist Silex für dieses
	Projekt hervorragend geeignet.
	
Für ein Szenario in der Realität kann Silex sehr gut eingesetzt werden solange das
	System einfach und klein ist. Mit mehr in dem Backendsystem implementierten
	Usecases sollte ein Wechsel zu Symfony2 in Betracht gezogen werden, da sich
	damit deutlich komplexere Anwendungsfälle implementieren lassen. Ein solcher
	Wechsel ist durch die bereits in Silex verwendeten Komponenten von Symfony2
	leicht zu vollziehen, die bereits bestehenden Komponenten können
	weiterverwendet werden.
	
In Kapitel~\ref{sec:webapplication} wird eine detaillierte des Backendsystems
	gegeben.

\subsection{Sensoren-Simulation mit Node.js}
\label{subsec:nodejs}

Für dieses Projekt wurden Temperatursensoren sowie Zeitsynchronisation mit
	Hilfe von \emph{Node.js}\footnote{vgl. \url{http://nodejs.org/}} simuliert.
	Es wurde in diesem Projekt das Augenmerk vor allem auf die mobile Applikation
	mit Android unter Verwendung einer verteilten Datenbank sowie der
	Webapplikation gelegt. Somit wurden	bis auf	Positionssensoren (GPS) die
	keine richtigen Sensoren verwendet.

Node.js ist ein ereignisgesteuertes I/O Framework für die V8 JavaScript
	Engine \cite{Wikipedia10a}. Diese wurde in C++ sowie JavaScript entwickelt
	und liegt in einer MIT-Lizenz vor, welches es für dieses Projekt einsetzbar
	macht und zugleich auch in einem realen Szenario eingesetzt werden könnte.

Mit Node.js können mit wenigen Zeilen Code, Server-Applikationen programmiert
	werden. Es wird hierzu auf einem Interface (IP) sowie einem beliebigen Port
	eine JavaScript Callback-Funktion registriert, welche bei einem Zugriff
	aufgerufen wird. Dies macht es sehr einfach, Sensoren in diesem System zu
	simulieren, welche via HTTP-Requests ``ausgelesen'' werden können.
	
In einem real logistischen System werden Sensoren nicht simuliert und somit
	spielt Node.js nur in diesem simulierten Szenario eine Rolle.
	
In Kapitel~\ref{sec:sensors} werden die in diesem System mit Node.js simulierten
	Sensoren erläutert. Zugleich werden die entsprechenden realen Sensoren
	beschrieben, welche in einer nicht simulierten Umgebung verwendet werden
	könnten.