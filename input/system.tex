\section{Systembeschreibung und -architektur}
\label{sec:system}

In diesem Abschnitt werden die in diesem Projekt verwendeten Technologien
	erläutert. Insbesondere werden die Gründe beschrieben, weswegen diese
	Technologien eingesetzt und anderen vorgezogen werden. Die Vor- sowie
	Nachteile der entsprechenden Technologien werden gegenübergestellt und
	besprochen. Zugleich werden die entsprechenden Technologien auf ihre
	Tauglichkeit in einem real logistischen Szenario geprüft.

\subsection{Mobiles Gerät auf Basis von Android}

Bei diesem Transportüberwachungssystem sollten laut den in
	Kapitel~\ref{sec:requirements} spezifizierten Anforderungen eine lückenlose
	und ständige Überwachung sichergestellt werden. Somit müssen Sensorendaten
	laufend auf ein Backendsystem übertragen werden. Hierzu bietet sich
	die \emph{Android}\footnote{vgl. \url{http://www.android.com/}} Plattform
	als Grundlage für eine mobile Applikation an.
	
Mit Android als Grundlage können gleich mehrere Aspekte abgedeckt werden. Jedes
	Fahrzeug wird mit einem Android Smartphone ausgestattet und ist somit nicht
	nur telefonisch erreichbar sondern gleichzeitig kann die komplette
	Transportüberwachung damit erreicht werden.
	
Mit der entwickelten Applikation lassen sich Gegenstände einladen, indem diese
	via Barcode in das System übernommen werden. Ab diesem Zeitpunkt wird dieser
	Gegenstand ständig überwacht und via CouchDB (siehe Abschnitt~\ref{subsec:couchdb})
	mit dem Backensystem synchronisiert. Auf der Webapplikation (siehe
	Abschnitt~\ref{subsec:webapplication}) können die Produkte bzw. die Lieferung,
	welche aus mehreren Produkten bestehen kann, nun auf einer Karte
	nachverfolgt werden, sowie die jeweiligen Daten der Temperatursensoren
	(siehe Abschnitt~\ref{subsec:nodejs}) bis zum Ausladen überprüft werden.
		
Auch in einem realen Szenario lässt sich Android hervorragend einsetzen. Es ist
	mittlerweile in Version~3.1 (15. Juni 2011) verfügbar und weit verbreitet.
	Die von Google bereitgestellten \emph{Google Apps for Business}\footnote{vgl.
	\url{http://www.google.com/apps/intl/de/business/index.html}} können die
	Smartphones per Fernwartung administriert werden. Dabei können auch
	Applikations-Updates an alle registrierten Smartphones verteilt werden.
	Somit ist auch eine einfache Lösung für das Deployment von neuen Versionen
	gegeben.
	
Im Kapitel~\ref{sec:android} wird die in diesem Projekt erstellte Android Applikation
	beschrieben.

\subsection{Verteiltes Datenbansystem CouchDB}
\label{subsec:couchdb}

Bei einem Transportüberwachungssystem ist die Datensicherung ein
	wichtiger Aspekt. In diesem Abschnitt wird die Datenverwaltung
	betrachtet und erläutert welches System für dieses
	verwendet wird.

In diesem Projekt wurde durch die in Kapitel~\ref{sec:requirements} spezifizierten
	Anforderungen ein Fokus auf die Verteiltheit des Systems gelegt. Es fallen
	durch die mobile Transportüberwachung Daten auf mobilen Geräten an, welche
	mit einem Backendsystem synchronisiert werden müssen. Aus diesem Grund wurde
	für das Datenbanksystem kein klassisches System in Betracht gezogen. Um einen
	neuen Ansatz in der Datenpersistierung zu erlernen, wurde das verteilte
	und dokumentbasierte Datenbankmanagementsystem \emph{CouchDB}\footnote{vgl.
	\url{http://couchdb.apache.org/}} verwendet.

In einem real logistischen Szenario muss auf die Skalierbarkeit sowie die
	Robustheit von CouchDB betrachtet werden. Klassische relationale
	Datenbankmanagementsysteme bringen durch die bereits sehr gut entwickelten
	Versionen vor allem einen Vorteil in der Robustheit und Stabilität. Dennoch,
	CouchDB wird bereits seit 2005 entwickelt und liegt aktuell in der stabilen
	Version 1.1.0 (6. Juni 2011) vor und wird bereits in mehreren kommerziellen
	Projekten wie beispielsweise in Ubuntu \cite{Murphy09}[S. 1] eingesetzt.

Einen detaillierten Überblick der Verwendung von CouchDB in diesem Projekt
	wird in Kapitel~\ref{sec:couchdb} gegeben.

\subsection{Webapplikation mit dem Framework Silex}
\label{subsec:webapplication}

Um ein benutzerfreundliches und einfaches Backendsystem
	für dieses Projekt zu erstellen, wurde auf mehrere bereits bestehende Frameworks
	zurückgegriffen. \emph{Silex}\footnote{vgl. \url{http://silex-project.org}} ist
	ein	Mikroframework für PHP 5.3. Es basiert wiederum auf dem Kern des
	\emph{Symfony2}\footnote{vgl. \url{http://symfony.com}} Frameworks.
	
Mit diesem Framework lassen sich einfache Webapplikationen sehr effizient in einer
	Model-View-Controller Umgebung implementieren. Durch die schöne Trennung der
	jeweiligen Schichten sowie der leichten Testbarkeit ist Silex für dieses
	Projekt hervorragend geeignet.
	
Für ein Szenario in der Realität kann Silex sehr gut eingesetzt werden, solange das
	System einfach und klein ist. Mit mehr in dem Backendsystem implementierten
	Anwendungsfällen sollte ein Wechsel zu Symfony2 in Betracht gezogen werden, da
	sich damit deutlich komplexere Anwendungsfälle implementieren lassen. Ein solcher
	Wechsel ist durch die bereits in Silex verwendeten Komponenten von Symfony2
	leicht zu vollziehen. Die bereits bestehenden Komponenten können
	weiterverwendet werden.
	
In Kapitel~\ref{sec:webapplication} wird eine detaillierte des Backendsystems
	gegeben.

\subsection{Sensoren-Simulation mit Node.js}
\label{subsec:nodejs}

Für dieses Projekt wurden Temperatursensoren sowie Zeitsynchronisation mit
	Hilfe von \emph{Node.js}\footnote{vgl. \url{http://nodejs.org/}} simuliert.
	Es wurde in diesem Projekt das Augenmerk vor allem auf die mobile Applikation
	mit Android unter Verwendung einer verteilten Datenbank sowie der
	Webapplikation gelegt. Somit wurden	bis auf	Positionssensoren (GPS) keine
	realen Sensoren verwendet.

Node.js ist ein ereignisgesteuertes I/O Framework für die V8 JavaScript
	Engine \cite{Wikipedia10a}. Diese wurde in C++ sowie JavaScript entwickelt
	und liegt in einer MIT-Lizenz vor, welches es für dieses Projekt einsetzbar
	macht und zugleich auch in einem realen Szenario eingesetzt werden könnte.

Mit Node.js können mit wenigen Zeilen Code, Server-Applikationen programmiert
	werden. Es wird hierzu auf einem Interface (IP) sowie einem beliebigen Port
	eine JavaScript Callback-Funktion registriert, welche bei einem Zugriff
	aufgerufen wird. Dies macht es sehr einfach, Sensoren in diesem System zu
	simulieren, welche via HTTP-Requests ``ausgelesen'' werden können.
	
In einem real logistischen System werden Sensoren nicht simuliert und somit
	spielt Node.js nur in diesem simulierten Szenario eine Rolle.
	
In Kapitel~\ref{sec:sensors} werden die in diesem System mit Node.js simulierten
	Sensoren erläutert. Zugleich werden die entsprechenden realen Sensoren
	beschrieben, welche in einer nicht simulierten Umgebung verwendet werden
	könnten.