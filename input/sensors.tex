\section{Transportüberwachung mittels Sensoren}\label{sensors}
\label{sec:sensors}

In diesem Projekt wird die Transportüberwachung mittels Sensoren, welche
	von der Android Applikation (siehe Kapitel~\ref{sec:android}) überwacht
	werden, realisiert.
	
Für die in diesem Projekt spezifizierten Anforderungen (siehe
	Kapitel~\ref{sec:requirements})
	wird die Temperatur sowie die aktuelle Position eines Gegenstands überwacht. Hierzu
	werden zwei unterschiedliche Sensortypen, welche in den beiden nachfolgenden Abschnitten
	erläutert werden, verwendet. Zusätzlich wird die Zeitsynchronisation der mobilen Geräte
	mittels eigens entwickelter Synchronisierung (wie in Abschnitt~\ref{subsec:timesync}
	erläutert) realisiert.

\subsection{Temperaturüberwachung}

Temperatursensoren werden in diesem Projekt simuliert. Alle benötigten
	Temperatursensoren werden mit \emph{node.js}, wie in
	Abschnitt~\ref{subsec:nodejs} erläutert, simuliert.

\subsection{Positionsüberwachung}

In diesem System werden in einem definierten Zeitintervall die aktuelle
	Position eines sich auf dem Transportweg befindenden Gegenstands aufgezeichnet.
	Die Positionsdaten werden von der Android Applikation bzw. der Service
	Applikation ausgelesen. Hierzu werden die Positionsdaten via GPS bzw. UMTS
	oder WLAN ermittelt und ebenfalls aufgezeichnet.
	
Die ermittelten Positionsdaten werden in der Webapplikation bei den Lieferungen
	jeweils auf einer Karte dargestellt.