\section{Webapplikation als Backendsystem}
\label{sec:webapplication}

Als unterstützendes Backendsystem wurde in diesem Projekt eine Webapplikation
	erstellt. Hiermit ist es möglich, eine Lieferung mit entsprechenden
	Gegenständen zu erstellen. Nach erfolgreichem Erstellen einer Lieferung
	kann diese auf einer Karte nachverfolgt werden. Gleichzeitig kann in
	einem Diagramm die Temperatur überwacht werden.

\subsection{Implementierung}

Das Backendsystem wurde in der Programmiersprache \emph{PHP}\footnote{vgl.
	\url{http://www.php.net}} implementiert. PHP ist eine dynamische
	Skriptsprache, die speziell für den Einsatz auf Webservern entwickelt
	wurde. Zum schnelleren Entwickeln, wurden mehrere Frameworks zur
	Unterstützung verwendet.
	
\subsubsection{Das Silex Framework}
\emph{Silex}\footnote{vgl. \url{http://silex-project.org/}} ist ein auf
	\emph{Symfony2}\footnote{vgl. \url{http://symfony.com/}} basierendes
	Mikro-Webapplikations-Framework für PHP 5.3. Es bietet eine überschaubare
	und intuitive API an, ist einfach zu erweitern und durchgängig mit
	Unit-Tests (PHPUnit\footnote{vgl. \url{http://www.phpunit.de/}}) getestet.
	
In diesem Projekt wurde eine Model-View-Controller Architektur (vgl.
	\cite{Schmidt09}[S. 354]) umgesetzt. Es wurde die Erstellung, Bearbeitung
	sowie Betrachtung von Lieferungen implementiert. Zusätzlich wurde eine
	Verwaltung für Transport-Einheiten (z.B. LKWs) und die Benutzerverwaltung
	für die mobile Applikation in die Webapplikation integriert.
		
\subsubsection{Doctrine2 ODM für CouchDB}
\emph{Doctrine2}\footnote{vgl. \url{http://www.doctrine-project.org/}} ist ein
	Framework zur Datenbankabstraktion. Im ursprünglichen Framework war nur ein
	\emph{Object-Relational-Mapper} (ORM) basierend auf dem
	\emph{Active-Record-Pattern} \cite{Schmidt09}[S. 380] vorhanden. Dadurch
	konnten nur relationale	Datenbanksysteme (z.B. Oracle oder MySQL) abstrahiert
	werden. Durch die Verwendung einer Dokumentenbasierten-Datenbank (sieht
	Kapitel~\ref{sec:couchdb}) wurde in diesem Projekt allerdings ein
	\emph{Object-Document-Mapper} (ODM) benötigt.
	
Dafür hat sich eine Erweiterung von Doctrine2 durch einen ODM für CouchDB
	angeboten, welcher sich allerdings erst in einem frühen Alpha-Stadion befindet.
	Dennoch wurde diese Erweiterung verwendet und in einigen Teilen sogar
	verbessert.

\subsubsection{JavaScript Framework jQuery}
Das JavaScript Framework \emph{jQuery}\footnote{vgl. \url{http://jquery.com/}}
	ist eine schnelle und einfach zu bedienende Bibliothek um in HTML-Dokument
	Manipulationen,	Ereignis-Behandlung, Animierung sowie Effekte und Ajax-
	Interaktionen durchzuführen.
	
jQuery wurde entwickelt um schneller Webapplikationen zu entwickeln. Es wurde der
	Fokus vor allem auf die Art, wie JavaScript in Webapplikationen verwendet wird,
	gelegt.
	
In diesem Projekt wurde die Darstellung von Graphen, Karten sowie die Validierung von
	Benutzereingaben mit jQuery realisiert.

\subsubsection{Blueprint CSS Framework}
Für die Erstellung eines einfachen \emph{Grid-Layout} \cite{W3C11}[S. 1] mithilfe von
	\emph{Cascading-Style-Sheets} (CSS) wurde in diesem Projekt das CSS-Framework
	\emph{Blueprint}\footnote{vgl. \url{http://www.blueprintcss.org/}} verwendet.
	
Hiermit lässt sich schnell ein grobes Grundgerüst für moderne Webapplikationen
	erstellen. Es verwendet eine ansprechende Typographie und bietet dem Designer
	bzw. Entwickler einen guten Ansatz für die Layoutgestaltung. Bei Bedarf kann
	dieses Framework mit einigen Plugins erweitert werden.

\subsection{Mögliche Erweiterungen}

Aus unternehmerischer Sicht wäre eine Anbindung an eine Kundendatenbank von
	großer Bedeutung. Durch diese können Lieferungen an wiederkehrende Auftraggeber
	einfacher eingetragen werden.
	
Eine andere wichtige Erweiterungsmöglichkeit wäre die Implementierung einer
	entsprechenden Zugriffskontrolle. Ein entsprechendes Rechtesystem für diese
	Webapplikation wäre vor allem in einem realen System von großer Bedeutung.

\subsection{Verwendung in einem realen System}

In einem realen System ist möglicherweise bereits eine Backend-Applikation
	vorhanden, welche um die entsprechenden Komponenten erweitert werden
	müsste. Die Backend-Datenbank (siehe Kapitel \ref{sec:couchdb}) kann
	auch an eine andere Applikation angebunden werden. Beispielsweise
	könnte eine Integration in eine SAP\footnote{vgl.
	\url{http://www.sap.com/germany/index.epx}} Umgebung erfolgen.
	
Sollten keine eigene Backend-Applikation im Unternehmen bestehen, sollte
	diese Webapplikation entsprechend erweitert werden. Beispielsweise
	muss eine Zugriffskontrolle, eine Kundendatenbank etc. implementiert
	werden.