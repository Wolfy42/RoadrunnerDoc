\section{Projektentwicklung}
\label{sec:development}

In diesem Projekt wurde ein agiler Softwareentwicklungsansatz gewählt. Es wurde
	die Entwicklung von Software in den Vordergrund gestellt und der klassischen,
	formalisierten Vorgehensweise geringe Bedeutung zugeteilt. Diese Vorgehensweise
	kann mit der von \cite{Beck98}[S. 25] et al. entwickelten Methode
	\emph{Extreme-Programming} verglichen werden.
	
Um diesen, durch fortlaufende Iterationen und den Einsatz mehrerer Einzelmethoden,
	sich stets ändernden Prozess erfolgreich anwenden zu können, ist eine entsprechende
	Disziplin sowie Kommunikationsbereitschaft im Team notwendig.
	
Durch den sogenannten \emph{Best-Practice} Ansatz kann in dieser Art der Entwicklung
	auf bereits vorhandene Lösungsansätze zurückgegriffen werden und somit der
	Entwicklungsprozess erheblich beschleunigt werden.
	
Da die jeweiligen Iterationen nur kleine Änderungen und in der Regel nur ein neues
	Feature in das System einführen, können auch sehr einfach neue Technologien
	getestet werden. Sollte es sich herausstellen, dass eine neue Technologie nicht
	die erwarteten Verbesserungen bringt, kann durch diese Vorgehensweise auch wieder
	schnell auf eine vorherige Iteration zurück gewechselt werden.
	
Konkret wurde in diesem Projekt in abwechselnden Zweierteams entwickelt und jedes
	neue Feature bzw. neuer Quellcode durch beide Entwickler besprochen und bei
	Bedarf verbessert.
	
Mit der Unterstützung eines verteilten Versionskontrollsystem (siehe
	Abschnitt~\ref{subsec:git}) konnte auch sehr einfach gleichzeitig an mehreren
	Features gearbeitet werden.