\section{Technologie}
\label{sec:technology}

In diesem Abschnitt werden die von diesem Projekt verwendeten Technologien erläutert. Insbesondere werden die Gründe erläutert, weswegen diese Technologien eingesetzt und anderen vorgezogen werden.

\subsection{CouchDB}
\label{subsec:couchdb}
CouchDB\footnote{The Apache CouchDB Project, \url{http://couchdb.apache.org/}} ist eine auf die Verwaltung von Dokumenten basierte Datenbank. CouchDB kann ähnlich wie das MapReduce Framework von Google\footnote{vgl. \url{http://en.wikipedia.org/wiki/MapReduce}} abgefragt sowie indiziert werden.

TODO: Ausführlicher beschreiben.

\subsection{Node.js}
\label{subsec:nodejs}

Node.js ist ein ereignisgesteuertes I/O Framework für die V8 JavaScript Engine \cite{Wikipedia10a}. Node.js wurde in C++ sowie JavaScript entwickelt und liegt in einer MIT-Lizenz vor, welches es für dieses Projekt einsetzbar macht und zugleich auch in einem real logistischem Szenario eingesetzt werden könnte.

Der Vorteil von \textit{nodejs} liegt darin, dass sehr schnell, mit wenig
Zeilen Code und Aufwand ein Http-Server programmiert werden kann.

\subsection{Benötigte Packete}
\begin{itemize}
  \item nodejs (für Unix)
  \item npm (um Bilbliothek optimist komfortabel zu installieren)
  \item optimist (nodejs Library für Console-Parameter übergabe)
\end{itemize}


\subsection{Silex/PHP}

TODO

\subsection{Android/Java}

TODO

\subsection{Barscanner}

TODO