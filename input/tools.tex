\section{Projektunterstützende Werkzeuge und Hilfsmittel}
\label{sec:tools}

In diesem Projekt wurden mehrere projektunterstützende Werkzeuge verwendet.
	Durch den sehr agilen Entwicklungsprozess (siehe Kapitel~\ref{sec:development})
	wurden vor allem auf eine entsprechende Versionskontrolle sowie ein
	Continuous-Integration-Server zur Überwachung des jeweiligen Entwicklungsstands
	geachtet. Diese beiden Systeme sind in den nachfolgenden Abschnitten erläutert.

\subsection{Versionskontrollsystem GIT mit github.com}
\label{subsec:git}

Git ist ein verteiltes Revisions-Kontroll-System mit einem Schwerpunkt auf
	Geschwindigkeit. Git wurde ursprünglich entworfen und
	entwickelt von Linus Torvalds für Linux-Kernel-Entwicklung \cite{Torvalds07}.
	Jedes Git-Arbeitsverzeichnis ist ein vollwertiges \emph{Repository} mit
	kompletter Historie und vollständige Überarbeitung Tracking-Fähigkeiten,
	nicht abhängig von Netzwerkzugang oder einen zentralen Server. Git aktuellen
	Software-Wartung wird durch Junio Hamano betreut. Git ist freie Software unter
	den Bedingungen der GNU General Public License Version 2 verteilt.
	
In diesem Projekt wurde Git für alle Teile des Projekts eingesetzt. Es wurde sowohl
	die CouchDB Applikation, die Android Applikation, die Webapplikation sowie die
	Dokumentation via Git verwaltetet und alle Projektmitglieder konnten in alle
	Teile einsehen sowie mitarbeiten.
	
Um Git auch Online zu synchronisieren, wurde in diesem Projekt
	gihub.com\footnote{vgl. \url{http://github.com}} verwendet. Via github ist es
	möglich, zusätzliche projektunterstützende Werkzeuge (wie z.B. Issue-Tracking,
	Wiki, etc.) zu verwenden.

\subsection{Continious Integration mit Jenkins}
\label{subsec:ci}

Jenkins, früher bekannt als Hudson bekannt, ist eine Open-Source-Continuous
	Integration (CI)-Tool, welches in Java geschrieben ist. Jenkins bietet
	kontinuierliche Integrations-Services für Software-Entwicklung an. Vor
	allem in der Programmiersprache Java. Es ist ein server-basiertes System
	mit einem Servlet-Container wie Apache Tomcat. Es unterstützt SCM-Tools
	einschließlich CVS, Subversion, Git und Clearcase und Apache Ant und Apache
	Maven basierende Projekte, sowie beliebige Shell-Skripte und
	Windows-Batch-Befehle auszuführen. Jenkins ist freie Software und wird
	unter der MIT-Lizenz veröffentlicht.
	
In diesem Projekt wurde sowohl die Android Applikation (Java) sowie die
	Webapplikation automatisiert via Jenkins getestet. Bei jedem neuen
	Commit, welcher zu github synchronisiert wurde, wird automatisiert
	ein Checkout von Jenkins in ein neues Verzeichnis erstellt. Danach
	wird das jeweilige Ant-Build-Script $build.xml$ ausgeführt und die
	Software kompiliert. Anschließend werden die definierten
	Analysewerkzeuge mit einer jeweiligen Konfiguration ausgeführt.
	
Bei der Android Applikation wurden JUnit\footnote{vgl.
	\url{http://www.junit.org/}} Tests ausgeführt.
	
Bei der Webapplikation wurden PHPUnit\footnote{vgl.
	\url{http://www.phpunit.de/}}, PHP-Check-Style, PHP-Mess-Detector sowie
	PHP-Copy-Paste-Detector verwendet um die Qualität zu überprüfen.
	

