\section{Applikationssicherheit}
\label{sec:security}

Die Anforderungen an die Sicherheit einer Applikation (Programm) sollen bereits
	zu Beginn der Entwicklung ermittelt und abgestimmt werden. Eine nachträgliche
	Implementierung von Sicherheitsmassnahmen ist bedeutend teurer und bietet im
	Allgemeinen weniger Schutz als Sicherheit, die von Beginn an in den
	Systementwicklungsprozess oder in den Auswahlprozess für ein Produkt
	integriert wurde.
	
In diesem Projekt werden die Zeitsynchronisierung sowie die Zugriffskontrolle für
	CouchDB unter dem Aspekt der Applikationssicherheit betrachtet.
	
Mit Hilfe der Zeitsynchronisierung wird die Systemzeit zwischen Android Smartphone
	und Backendsystem abgeglichen um eine einheitliche Zeit gewährleisteten zu können.

Die Zugriffskontrolle für CouchDB dient als Sicherheitsmechanismus um die Daten vor
	Zugriffen Dritter zu schützen.

\subsection{Zeitsynchronisierung}
\label{subsec:timesync}

In diesem Abschnitt werden Probleme besprochen, die durch fehlerhafte respektive
	mangelhafte Zeitsynchronisierung entstehen können.

\paragraph{Problem durch falsche Zeitstempel bei Logeinträgen} Betrachtet wird das
	Szenario ``Umladevorgang eines Produktes''. Das mobile Gerät der Transporteinheit
	wird benutzt um den Ausladevorgang aus einem Container im System zu verarbeiten.
	Mit dem Scannen des Produkts wird auf dem mobilen Gerät der Transporteinheit ein
	Logeintrag in dessen lokale Datenbank erstellt. Genauso wird beim darauffolgenden 
	Ladevorgang der Umladestation ein Logeintrag auf dessen Gerät erstellt. Wenn das
	System mit absoluter Zeit arbeitet und die Uhrzeit des Geräts der Transporteinheit
	vor jener der Umladestation ist, dann würde im System der Übernahmevorgang der
	Umladestation vor dem Ausladevorgang der Transporteinheit stattfinden.


\paragraph{Lösungsansatz} Um dieses Problem zu lösen muss relative Zeit eingeführt und
	synchronisiert werden. Für die Zeitsynchronisierung können bekannte Algorithmen
	für verteilte Systeme verwendet werden. Mögliche Algorithmen sind Christian's
	Algorithmus \cite{Christian89}[S. 146-158] oder Berkley Algorithmus\footnote{vgl.
	\url{http://ieeexplore.ieee.org/xpl/freeabs_all.jsp?&arnumber=29484}}.

In diesem Projekt wurde Christian's Algorithmus implementiert. Jeder Client misst seine
	Differenz zur Serverzeit und verwendet diese zur Erzeugung der Zeitstempel. Somit
	können die unterschiedlichen Uhren bis zu einer gewünschten Genauigkeit
	synchronisiert werden.

Die Systemzeit wird bis zu einer gewählten Genauigkeit synchronisiert.

\subsection{Zugriffskontrolle}

Die Benutzerrechte in diesem Projekt werden mit verschiedene Benutzergruppen abgebildet.
	Die Benutzerauthentifizierung wird von CouchDB durchgeführt.

\subsubsection{Administratoren und System-Benutzer}

Auf einer Datenbank können in CouchDB Administratoren und normale Benutzer definiert
	werden.	
	
\paragraph{Administrator} Ein Administrator hat sämtliche Rechte auf der Datenbank.
	Er kann die Datenbank löschen, Designdokumente verändern oder Benutzerrechte ändern.
	
\paragraph{Benutzer} Ein Benutzer hat lesenden und schreibenden Zugriff auf
	alle Dokumente bis auf die Designdokumente.

\subsubsection{Rollen im System}

Einem Benutzer können keine bis mehrere Rollen zugewiesen werden. Bei jeder Veränderung
	von Dokumenten wird von CouchDB eine Benutzerauthentifizierung
	durchgeführt. Bei dieser Benutzerauthentifizierung wird das Zugriffsrecht auf die
	Datenbank überprüft und zudem eine Validierung durchgeführt. Bei der Validierung
	werden alle definierten Validierungsmethoden aufgerufen. Nur wenn alle Validierungen
	gültig sind wird die gewünschte Änderung an den Dokumenten durchgeführt.