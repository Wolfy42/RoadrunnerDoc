\section{Applikationssicherheit}
\label{sec:security}

TODO: einführender absatz in app sicherheit

\subsection{Zeitsynchronisierung}
\label{subsec:timesync}

In diesem Abschnitt werden Probleme besprochen, die durch fehlerhafte respektive
	mangelhafte Zeitsynchronisierung entstehen können.

\paragraph{Problem durch falsche Zeitstempel bei Logeinträgen} Betrachtet wird das
	Szenario ``Umladevorgang eines Produktes''. Das mobile Gerät der Transporteinheit
	wird benutzt um den Ausladevorgang aus einem Container im System zu verarbeiten.
	Mit dem Scannen des Produkts wird auf dem mobilen Gerät der Transporteinheit ein
	Logeintrag in dessen lokale Datenbank erstellt. Genauso wird beim darauffolgenden 
	Ladevorgang der Umladestation ein Logeintrag auf dessen Gerät erstellt. Wenn das
	System mit absoluter Zeit arbeitet und die Uhrzeit des Geräts der Transporteinheit
	vor jener der Umladestation ist, dann würde im System der Übernahmevorgang der
	Umladestation vor dem Ausladevorgang der Transporteinheit stattfinden.


\paragraph{Lösungsansatz} Um dieses Problem zu lösen muss relative Zeit eingeführt und
	synchronisiert werden. Für die Zeitsynchronisierung können bekannte Algorithmen
	für verteilte Systeme eingeführt werden. Mögliche Algorithmen sind Christian's
	Algorithmus \cite{Christian89}[S. 146-158] oder Berkley Algorithmus\footnote{vgl.
	\url{http://ieeexplore.ieee.org/xpl/freeabs_all.jsp?&arnumber=29484}}.

In diesem Projekt wurde Christian's Algorithmus implementiert. Jeder Client misst seine
	Differenz zur Serverzeit und verwendet diese zur Erzeugung der Zeitstempel. Somit
	können die unterschiedlichen Uhren bis zu einer gewünschten Genauigkeit
	synchronisiert werden.

Grundsätzlicher Ablauf zur Ermittlung der Zeitdifferenz ist folgender:

\begin{enumerate}
	\item Client erzeugt einen Zeitstempel mit der letzten bekannten
		Differenz zur Serverzeit.
	\item Client sendet eine Anfrage für die Serverzeit an den Server.
	\item Server antwortet mit seiner Zeit.
	\item Client erzeugt einen zweiten Zeitstempel.
	\item Ist die RoundTrip-Zeit klein genug werden die Zeitstempel
		von Client und Server verglichen.
	\item Überschreitet die Zeitdifferenz einen Schwellwert wird auf
		dem Client die neue Zeitdifferenz gesetzt.
\end{enumerate}

Die Systemzeit werden bis zu einer gewählten Genauigkeit miteinander synchronisiert.
	In diesem Projekt wurde die Genauigkeit mit 5 Sekunden gewählt. Dies bedeutet,
	wenn der Client seinen Zeitstempel mit dem von dem Server vergleicht und die
	Differenz die Genauigkeit überschreitet, dann ermittelt der Client die neue
	Differenz und verwendet diese zur Erzeugung der nächsten Zeitstempel. Die
	Synchronisation bis zu einer Genauigkeit von 5 Sekunden wurde gewählt, da ein
	Umladevorgang länger als fünf Sekunden dauert. Eine zu kleine Genauigkeit
	hätte zur Folge, dass durch Abweichende RoundTrip-Zeiten ständig die Uhr neu
	gestellt würde und unnötig viele Zeitsynchronisierungs-Einträge erstellt würden.
	
Bei jeder Zeitsynchronisierung wird eine Log-Eintrag für alle geladenenen Produkte erzeugt.
	Dadurch ist in der History eines Produkts ersichtlich wann eine Zeitsynchronisierung
	des Client durchgeführt wurde und welche Log-Einträge einen von der Serverzeit
	abweichenden Zeitstempel haben.

Bei der Zeitsynchronisierung ist die RoundTrip-Zeit ebenfalls von großer Bedeutung.
	Ist die RoundTrip-Zeit zu groß wird der Zeitstempel vom Server verworfen und keine
	Synchronisierung durchgeführt. Nur wenn die RoundTrip-Zeit hinreichend klein ist,
	kann eine sinnvolle Zeitdifferenz der Zeitstempel ermittelt werden. Für die größte
	noch erlaubte RoundTrip-Zeit wurde zwei Sekunden gewählt. Dies entspricht etwa
	der halben Genauigkeit und bei keiner zu großen Netzauslastung sollte eine Anfrage
	in dieser Zeit auch bearbeitbar sein.

\subsection{Zugriffskontrolle}

Zur Umsetzung des Rechtesystems von Roadrunner werden verschiedene Benutzergruppen
	eingeführt. Die Rechte werden einerseits direkt auf der Datenbank definiert und
	zudem noch über Validierungsfunktionen umgesetzt. Die Benutzerauthentifizierung
	wird von CouchDB durchgeführt.

\subsubsection{Administratoren und System-Benutzer}

Auf einer Datenbank können in CouchDB Administratoren und normale Benutzer definiert
	werden.
	
\begin{description}
	\item[Administrator] Ein Administrator hat sämtliche Rechte auf der Datenbank.
		Er kann die Datenbank löschen, Designdokumente verändern oder
		Benutzerrechte ändern.
	\item[Benutzer] Ein Benutzer hat lesenden und schreibenden Zugriff auf
		alle Dokumente bis auf die Designdokumente.
\end{description}

\noindent Ein Benutzer kann über 2 verschiedene Arten einer Gruppe zugeordnet werden:

\begin{itemize}
	\item[Namen] Ein CouchDB-Benutzer muss einen eindeutigen Namen im Format
		$org.couchdb.user:[username]$ haben. Die Benutzer werden in einer separaten
		Datenbank mit dem Namen $\_users$ definiert. Wenn der Benutzer der Namensliste
		hinzugefügt wird, dann besitzt er die entsprechenden Rechte.
		
	\item[Rollen] Ein Benutzer kann verschiedene Rollen besitzen. Wenn eine seiner
		Rollen in der Liste der Rollen aufgeführt wird, hat er die entsprechenden
		Rechte.
\end{itemize}

\subsubsection{Rollen in diesem System}

Einem Benutzer können keine bis mehrere Rollen zugewiesen werden. Bei jeder Veränderung
	von Dokumenten wird von CouchDB eine Benutzerauthentifizierung
	durchgeführt. Bei dieser Benutzerauthentifizierung wird das Zugriffsrecht auf die
	Datenbank überprüft und zudem eine Validierung durchgeführt. Bei der Validierung
	werden alle definierten Validierungsmethoden aufgerufen. Nur wenn alle Validierungen
	gültig sind wird die gewünschte Änderung an den Dokumenten durchgeführt.
	
\noindent Im diesem Projekt wurden drei verschiedenen Rollen definiert:

\begin{description}
	\item[Admin] Ein Administrator hat sämtliche Rechte auf der Datenbank. Diese
		Rolle ist ausschließlich für Administratoren vorgesehen.
	\item[Office] Ein Benutzer der Gruppe Office arbeitet mit dem Backendsystem von
		Roadrunner. Dieser Benutzer arbeitet über die Webapplikation mit diesem System.
	\item[Driver] Ein Benutzer der Gruppe Driver ist ein Fahrer. Er arbeitet auf dem
		Androidsystem mit der Roadrunner Applikation. Auf dem Androidsystem arbeitet
		er als Admin mit der Datenbank. Eine Einschränkung der Benutzerrechte auf der
		Android Plattform ist nicht notwendig, da die Benutzerauthentifizierung bei der
		Replizierung der Daten von dem Androidsystem auf das Backendsystem durchgeführt
		wird. Ein Fahrer kann nur Daten replizieren für die er die entsprechenden Rechte
		besitzt.
\end{description}

\noindent In den Validierungsmethoden werden die entsprechenden Rechtevalidierungen
	durchgeführt. In Tabelle \ref{tab:rechte} sind die Berechtigungen aufgelistet.
	Ein \textbf{+} bedeutet, dass der Benutzer die entsprechende Berechtigung besitzt.

\begin{table}[h]
	\begin{tabular}{lccc}
		& \textbf{Driver} & \textbf{Office} & \textbf{Admin} \\
		Benutzerrechte ändern & - & - & + \\
		Designdokumente ändern & - & - & + \\
		Logeinträge anlegen & + & + & + \\
		Logeinträge ändern & - & - & + \\
		Logeinträge löschen & - & - & + \\
		Andere Dokumente anlegen & - & + & + \\
		Andere Dokumente ändern & - & + & + \\
		Andere Dokumente löschen & - & + & +
	\end{tabular}
	\caption{Benutzerrechte}
	\label{tab:rechte}
\end{table}
