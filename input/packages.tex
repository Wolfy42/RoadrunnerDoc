%%%%%%%%%%%%%%%%%%%%%%%%%%%%%%%%%%%%%%%%%%%%%%%%%%%%%%%%%%%%%%%%%%%%%
% Packages

% für \hyphenation mit Umlauten
\usepackage[T1]{fontenc}
\usepackage[utf8]{inputenc}
\usepackage[ngerman,english]{babel}

% Times-Roman-Schrift (auch für mathematische Formeln)
\usepackage{mathptmx} 

% comments
\usepackage{verbatim} 

% Zum Setzen von URLs
\usepackage{color}
\usepackage{alltt}
\definecolor{darkred}{rgb}{.25,0,0}
\definecolor{darkgreen}{rgb}{0,.2,0}
\definecolor{darkmagenta}{rgb}{.2,0,.2}
\definecolor{darkcyan}{rgb}{0,.15,.15}

\usepackage[plainpages=false,bookmarks=true,bookmarksopen=true,colorlinks=true,
  linkcolor=darkred,citecolor=darkgreen,filecolor=darkmagenta,
  menucolor=darkred,urlcolor=darkcyan]{hyperref}

% Zeilenabstand
\renewcommand{\baselinestretch}{1.5}

% anhang
\usepackage[toc,page]{appendix}

% pdflatex: Bilder in den Formaten .jpeg, .png und .pdf
% latex: Bilder im .eps-Format
\usepackage{graphicx}

\usepackage{fancyhdr} % Positionierung der Seitenzahlen
\fancyhead{}
\fancyfoot[C]{\Roman{page}}
\renewcommand{\headrulewidth}{0pt}
\setlength{\headheight}{13.6pt} % behebt headheight Warning

 % behebt headheight Warning
\setlength{\headheight}{13.6pt}

% Korrektes Format für Nummerierung von Abbildungen (figure) und
% Tabellen (table): <Kapitelnummer>.<Abbildungsnummer>
\makeatletter
\@addtoreset{figure}{section}
\renewcommand{\thefigure}{\thesection.\arabic{figure}}
\@addtoreset{table}{section}
\renewcommand{\thetable}{\thesection.\arabic{table}}
\makeatother

% Listings für Sourcecode
\usepackage{listings}
  \usepackage{courier}
 \lstset{
        basicstyle=\footnotesize\ttfamily, % Standardschrift
        numbers=left,               % Ort der Zeilennummern
        numberstyle=\tiny,          % Stil der Zeilennummern
        %stepnumber=2,               % Abstand zwischen den Zeilennummern
        numbersep=5pt,              % Abstand der Nummern zum Text
        tabsize=2,                  % Groesse von Tabs
        extendedchars=true,         %
        breaklines=true,            % Zeilen werden Umgebrochen
        keywordstyle=\color{red},
        frame=b,         
        keywordstyle=[1]{\color{DarkSkyBlue}},    % Stil der Keywords
        keywordstyle=[2]{\color{DarkScarletRed}},    %
        keywordstyle=[3]{\bfseries},    %
        keywordstyle=[4]{\color{DarkPlum}},    %
        keywordstyle=[5]{\color{SkyBlue}},    %
		stringstyle={\color{Chocolate}},
        showspaces=false,           % Leerzeichen anzeigen ?
        showtabs=false,             % Tabs anzeigen ?
        xleftmargin=17pt,
        framexleftmargin=17pt,
        framexrightmargin=5pt,
        framexbottommargin=4pt,
        backgroundcolor=\color{Aluminium1},
        showstringspaces=false,      % Leerzeichen in Strings anzeigen ?
		%language=php
		morekeywords=[1]{Interface,return,static,function}
}
    %\DeclareCaptionFont{blue}{\color{blue}} 

  %\captionsetup[lstlisting]{singlelinecheck=false, labelfont={blue}, textfont={blue}}
  \usepackage{caption}
\DeclareCaptionFont{white}{\color{white}}
\DeclareCaptionFormat{listing}{\colorbox[cmyk]{0.43, 0.35, 0.35,0.01}{\parbox{\textwidth}{\hspace{15pt}#1#2#3}}}
\captionsetup[lstlisting]{format=listing,labelfont=white,textfont=white, singlelinecheck=false, margin=0pt, font={bf,footnotesize}}
\renewcommand\lstlistingname{Codeblock}
 

\sloppy % Damit LaTeX nicht so viel über "overfull hbox" u.Ä. meckert

% Ränder
\addtolength{\topmargin}{-16mm}
\setlength{\oddsidemargin}{40mm}
\setlength{\evensidemargin}{40mm}
\addtolength{\oddsidemargin}{-1in}
\addtolength{\evensidemargin}{-1in}
\setlength{\textwidth}{13cm}
\addtolength{\textheight}{34mm}
%______________________________________________________________________

% Verhindert Hurenkinder & Schusterjungen
\clubpenalty = 10000
\widowpenalty = 10000
\displaywidowpenalty = 10000

