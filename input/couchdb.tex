\section{CouchDB Applikation}
\label{sec:couchdb}

Apache CouchDB ist ein Dokument-Orientiertes-Datenbanksystem für die Verwendung
	mit JavaScript. CouchDB bietet inkrementelle Replikation mit bi-directionaler
	Konflikt-Erkennung und -Lösung.
	
CouchDB bietet eine REST-API \cite{Fowler10}[S. 1] via \emph{JavaScript Object
	Notation} (JSON) an, welche von jeder beliebigen Umgebung mit Hilfe von
	HTTP-Requests abgerufen werden kann. CouchDB ist zusätzlich ein System, welches
	eine beliebige Skalierbarkeit sowie Erweiterbarkeit anbietet
	\cite{CouchDB11}[S. 1].
	
In diesem Projekt wurde CouchDB eingesetzt, um ein relativ neues Gebiet der
	Datenpersistierung zu erlernen. Durch die einfache Replizierung von Daten,
	konnte CouchDB sowohl auf der Backend-Webapplikation (vgl.
	Kapitel~\ref{sec:webapplication}) als auch auf den mobilen Geräten (vgl.
	Kapitel~\ref{sec:android}) eingesetzt werden.

\subsection{Implementierung}

\subsubsection{JSON und Schema-Validierung}
Als Dokumentstruktur wird von CouchDB JSON verwendet. Vor der Speicherung eines
	JSON-Dokumentes in die Datenbank werden die Validierungsmethoden von allen
	Designdokumenten in der Datenbank aufgerufen. Nur wenn alle Validierungen
	erfolgreich sind wird das Dokument gespeichert. Obwohl JSON Schemalos ist
	kann trotzdem eine Schemavalidierung durchgeführt werden. Als Schema wird
	das JSON-Schema verwendet, dass sich aktuell in Version 03
	befindet \cite{IETF11}[S. 1]. 

In einem Designdokument können verschiedene Validierungen eingeführt werden.
	Zusätzlich zu Validierungen der Benutzerrechte werden in diesem Projekt alle
	Dokumente auf das definierte JSON-Schema validiert.

\subsection{Datenverteilung}

Daten können auf unterschiedliche Arten in einem System verteilt werden. Eine
	Möglichkeit ist, die Daten aus der mobilen Datenbankinstanz in die Applikation
	zu lesen und auf der Applikationsschicht die anfallenden Daten an die
	Master-Datenbank zu senden.

Mit CouchDB wird die Synchronisierung der Daten in diesem Projekt von den
	Datenbankinstanzen selbst durchgeführt. Diese Synchronisierung nennt sich 
	Replizierung. Bei dieser kann explizit angegeben werden, welche Daten
	in jeweils welche Richtung synchronisiert werden.

\subsubsection{Dokumentänderung - MapReduce}

\emph{MapReduce}\footnote{vgl. \url{http://labs.google.com/papers/mapreduce-osdi04.pdf}}
	ist ein Framework von Google, dass entwickelt wurde damit sehr große
	Datenmengen parallel bearbeitet werden können. CouchDB verwendet ebenfalls
	einen ähnlichen Ansatz um Daten aus der Datenbank zu lesen. Anhand eines
	Beispiels wird die Funktionsweise von MapReduce nachfolgend erläutert.

\paragraph{Map - Phase} Auf jedes Dokument in der Datenbank wird die Map-Methode
	angewendet. In einer Map-Methode werden Key-Value-Paarungen gebildet. Jedes
	Dokument in der Datenbank kann eine beliebige Anzahl an Key-Value-Paarungen
	generieren. Diese Key-Value-Paarungen werden in einem B-Baum (vgl.
	\cite{Ottmann96}[S. 317-327]) gespeichert. Ändert sich nun ein Dokument müssen
	nur die entsprechenden Paarungen in	dem B-Baum angepasst werden. 

\paragraph{Reduce - Phase} In dieser Phase wird auf jedem Knoten im Baum die
	Reduce-Methode angewendet. Ziel der Reduce-Methode ist es die Datenmenge zu
	minimieren.

\subsection{Mögliche Erweiterungen}

Mögliche Erweiterungen wären zusätzliche Views mit denen bestimmte Informationen
	aus der Datenbank gelesen werden können. In diesem Projekt ist eine View
	umgesetzt, welche den Status eines Produkts (geladen, gelifert, etc.) ermitteln
	kann.

Wichtig wäre auch ein View, die den Status einer ganzen Lieferung ermittelt oder
	eine View zur Ermittlung an welchen Positionen sich die einzelnen Fahrzeuge
	gerade befinden, sodass die Fahrzeugposition beispielsweise auf einer Karte
	dargestellt werden kann.

\subsection{Verwendung in einem realen Projekt}

Bei der Umsetzung in einem realen System liegen möglicherweise bereits Daten des
 	Unternehmens vor. Da relationale Datenbanken weit verbreitet sind, müssen die
	bereits vorhandenen Daten per Importer in CouchDB übertragen werden.

Weiters muss die Datenbank per Proxy abgesichert werden, sodass über das Internet
	nur über klar definierte Schnittstellen auf die Datenbank zugegriffen werden
	kann. Die externe Erreichbarkeit der Datenbank ist notwendig, da die mobile
	Android-Applikation die Datenbank erreichen können muss um Dokumente zu replizieren.