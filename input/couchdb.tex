\section{CouchDB Applikation}
\label{sec:couchdb}

Apache CouchDB ist ein Dokument-Orientiertes-Datenbanksystem für die Verwendung
	mit JavaScript. CouchDB bietet inkrementelle Replikation mit bi-directionaler
	Konflikt-Erkennung und -Lösung.
	
CouchDB bietet eine REST-API \cite{Fowler10}[S. 1] via \emph{JavaScript Object
	Notation} (JSON) an, welche von jeder beliebigen Umgebung mit Hilfe von
	HTTP-Requests abgerufen werden kann. CouchDB ist zusätzlich ein System, welches
	eine beliebige Skalierbarkeit sowie Erweiterbarkeit anbietet
	\cite{CouchDB11}[S. 1].
	
In diesem Projekt wurde CouchDB eingesetzt, um ein relativ neues Gebiet der
	Datenpersistierung zu erlernen. Durch die einfache Replizierung von Daten,
	konnte CouchDB sowohl auf der Backend-Webapplikation (vgl.
	Kapitel~\ref{sec:webapplication}) als auch auf den mobilen Geräten (vgl.
	Kapitel~\ref{sec:android}) eingesetzt werden.

\subsection{Implementierung}

TODO: roadrunner.server

\subsubsection{JSON und Schema-Validierung}
Als Dokumentstruktur wird von CouchDB JSON verwendet. Vor der Speicherung eines
	JSON-Dokumentes in die Datenbank werden die Validierungsmethoden von allen
	Designdokumenten in der Datenbank aufgerufen. Nur wenn alle Validierungen
	erfolgreich sind wird das Dokument gespeichert. Obwohl JSON Schemalos ist
	kann trotzdem eine Schemavalidierung durchgeführt werden. Als Schema wird
	das JSON-Schema verwendet, dass sich aktuell in Version 03
	befindet \cite{IETF11}[S. 1]. 

In einem Designdokument können verschiedene Validierungen eingeführt werden.
	Zusätzlich zu Validierungen der Benutzerrechte werden in diesem Projekt alle
	Dokumente auf das definierte JSON-Schema validiert.

\subsubsection{Designdokumente}

TODO

\subsubsection{Dokumentänderung - MapReduce}

\emph{MapReduce} ist ein Framework von Google, dass entwickelt wurde damit sehr große
	Datenmengen parallel bearbeitet werden können. CouchDB verwendet ebenfalls
	einen ähnlichen Anstaz um Daten aus der Datenbank zu lesen. Anhand eines
	Beispieles wird die Funktionsweise von MapReduce nachfolgend erläutert.

Das Beispiel beantwortet folgende Problemstellung: Welche Gegenstände wurden
	von einer Transporteinheit (durch das Lesen des Barcodes) eingeladen
	und sind somit in der Datenbank als geladen gekennzeichnet?

\paragraph{Map - Phase} Auf jedes Dokument in der Datenbank wird die Map-Methode
	angewendet. In einer Map-Methode werden Key-Value-Paarungen gebildet. Jedes
	Dokument in der Datenbank kann eine beliebige Anzahl an Key-Value-Paarungen
	generieren. Diese Key-Value-Paarungen werden in einem B-Baum (vgl.
	\cite{Ottmann96}[S. 317-327]) gespeichert. Ändert sich nun ein Dokument müssen
	nur die entsprechenden Paarungen in	dem B-Baum angepasst werden. 

\paragraph{Reduce - Phase} In dieser Phase wird auf jeden Element in dem Baum die
	Reduce-Methode angewendet. Ziel der Reduce-Methode ist es die Datenmenge zu
	minimieren. Auf jedes Element kann die Reduce-Methode beliebig oft angewendet
	werden.

\subsection{Mögliche Erweiterungen}

TODO

\subsection{Verwendung in einem realen Projekt}

TODO