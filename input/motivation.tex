\section{Motivation}

Ein Transportüberwachungssystem kann aus mehreren Blickwinkeln betrachtet werden.
	In diesem Semesterprojekt ist für die Erstellung eines vollständigen
	Systems zu wenig Zeit vorhanden und somit wird in diesem Projekt der Fokus
	auf die Erstellung einer mobilen Applikation, die Replizierung von Daten auf
	einen Backendserver sowie die Verwaltung des Systems mit einer Webapplikation
	gelegt.
	
Aus oben genannten zeitlichen Gründen wird keine eigene Hardware entwickelt und somit
	die Android Plattform als Host-System für eine mobile Applikation verwendet.
	Zudem werden bis auf eine Ausnahme keine realen Sensoren verwendet sondern
	benötigte Sensoren simuliert.

Dieses Projekt konzentriert sich auf die Entwicklung von Software und wir mit Hilfe
	neuer Technologien, teilweise sogar in Beta-Versionen, implementiert.
	
Auf dem Backendserver wird mit CouchDB ein unkonventioneller Ansatz der
	Datenpersistierung gewählt. Zur Verwaltung von Aufträgen (Lieferungen)
	dient eine Webapplikation.

Als Softwareentwicklungsprozess wird \emph{Test-Driven-Development} gewählt.
	Hierzu wurden nahezu alle entwickelten Komponenten mit \emph{Unit-Tests}
	getestet sowie wenn möglich auf einem \emph{Continuous-Integration}-Server
	bei jeder Änderung automatisiert getestet.